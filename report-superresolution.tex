\documentclass{paper}

%\usepackage{times}
\usepackage{epsfig}
\usepackage{graphicx}
\usepackage{amsmath}
\usepackage{amssymb}
\usepackage{color}
\usepackage{hyperref}


% load package with ``framed'' and ``numbered'' option.
%\usepackage[framed,numbered,autolinebreaks,useliterate]{mcode}

% something NOT relevant to the usage of the package.
\setlength{\parindent}{0pt}	
\setlength{\parskip}{18pt}
\newcommand{\norm}[1]{\left\lVert#1\right\rVert}

\usepackage[latin1]{inputenc} 
\usepackage[T1]{fontenc} 

\usepackage{listings} 
\lstset{% 
   language=Matlab, 
   basicstyle=\small\ttfamily, 
} 



\title{Assignment 1 - Superresolution}



\author{Moser Stefan\\09-277-013}
% //////////////////////////////////////////////////


\begin{document}



\maketitle


% Add figures:
%\begin{figure}[t]
%%\begin{center}
%\quad\quad   \includegraphics[width=1\linewidth]{ass2}
%%\end{center}
%
%\label{fig:performance}
%\end{figure}

\section{Problem}
Through various effects, an image may end up with a low resolution. 
This can be due to hardware limitations 
(e. g. an image taken from a cheap security camera), 
bandwith limitations 
(e. g. an image was downscaled for transmitting over a connection with very low bandwith) or others.  


\section{Motivation}
To retrieve further details, we might now want to increase image quality by increasing its resolution. 
While this is not possible in a manner as depicted in \href{http://petapixel.com/2012/08/17/ridiculous-photo-enhancement-scene-from-the-tv-show-csi/}{popular TV series}, 
we can make certain assumptions for regularization combined with convex optimization to increase quality by a fair amount. 
In this project use a regularization term that penalizes high gradients,
assuming that the original image largely consists of smooth patches.

\section{Derivation of Gradient}
\begin{equation}
\min_{u \in X} \norm{\nabla u}_1 + \frac{\lambda}{2} \norm{Du - g}^2_2
\end{equation}

Discretized 
\begin{equation}
\min_{u} \sum_{i=1}^{N-1} \norm{u[i+1] - u[i]}_2 + 
\frac{\lambda}{2} \sum_{i=1}^{N-1} (Du[i] - g[i])^2
\end{equation}

Finite gradient:
TODO: show that chain rule does eliminate square root.

\begin{equation}
\nabla E[i] = 2 (2u[i] - u[i + 1] - u[i - 1]) + \lambda(Du[i] - g[i])
\end{equation}
\begin{enumerate}
\item \textbf{Derivation of gradient.} In this section you should:

\begin{itemize}
\item Write the finite difference approximation of the objective function $E$.
\item Compute the gradient of the objective function $\nabla_uE$.  
\end{itemize}


\item \textbf{Implement gradient descent for superresolution.} In this section you should:

\begin{itemize}
\item Show some images, as the the gradient method progresses iteration by iteration. Display the initial and the final image and 3 more images in between.
\end{itemize}

\item \textbf{Show images obtained by very high, very low and optimal $\lambda$.} In this section you should:

\begin{itemize}
\item Display 3 images with different $\lambda$ (very low, very high and optimal).
\item Describe the effect of $\lambda$ on the solution.
\end{itemize}

\item \textbf{ Find optimal $\lambda$.} In this section you should:

\begin{itemize}
\item Display the $SSD$ vs. $\lambda$ graph.
\item Describe the effect of $\lambda$ with respect to the $SSD$ between the ground truth and the solution image.
\end{itemize}


\end{enumerate}


 \end{document}
 
 